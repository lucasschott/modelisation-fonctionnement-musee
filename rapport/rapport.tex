\documentclass[a4paper,12pt,DIV=16]{scrreprt}
\usepackage[utf8]{inputenc}
\usepackage[T1]{fontenc}
\usepackage[francais]{babel}

%opening
\title{Rapport de TP:\@ TP3 modélisation du fonctionnement d'un musée}
\author{Lucas SCHOTT}

\begin{document}


    \maketitle


    \section*{Introduction}
     
    \\

    Le but de ce TP est de modéliser le fonctionnement d'un musée à l’aide de
    processus, matérialisés par des commandes indépendantes travaillant sur un
    ensemble d’objets IPC System V (mémoire partagée et sémaphores).\\

    Cette modélisation est composé de trois entitées qui effectuent différentes
    actions:\\
    Il y a un directeur qui peut créer un musée, l'ouvrir, le fermer
    et le supprimer.\\
    Un contrôleur qui contrôle les entrées et sorties des
    visiteurs pour éviter qu'il n'y ai trop de visiteurs en même temps dans le
    musée.\\
    Et des visiteurs qui viennent visiter le musée.\\ \\


    Le modèle fonctionne de la manière suivante:
    \begin{itemize}
        \item   Le directeur crée un musée, puis peut l'ouvrir et le fermer
            autant de fois qu'il veut, puis il peut le supprimer.
        \item   Le contrôleur doit être en poste avant l'ouverture du musée,
            une fois le musée ouvert, il contrôle les entrées et sorties des
            visiteurs. Il fait entrer dans le musée un visiteur qui est dans le
            file d'attente uniquement s'il y a de la place dans le musée. Quand
            le musée ferme, il arrête de faire entrer des visiteurs, et attend
            que tous les visiteur déjà dans le musée soient sortie. Une fois
            qu'il n'y a plus de visiteurs dans le musée, le contrôleur quitte
            son poste (il faut mettre un nouveau contrôleur en poste avant une
            réouverture du musée).
        \item   Le visiteur arrive aux abords du musée. Si le musée est fermé
            le visiteur attend son ouverture. Quand le musée est ouvert le
            visiteur regarde la file d'attente, si la file d'attente est
            pleine, le visiteur s'en va, alors que s'il ya de place dans la
            file, le visiteur y prend place. Une fois dans la file, le visiteur
            demande au contrôleur l'autorisation d'entrer dans le musée. Cette
            autorisation lui est accordé une fois qu'il y a de la place dans le
            musée. Une fois dans le musée, le visiteur y reste un certain
            temps, puis en sort quand il a fini de visiter.
    \end{itemize}


    \section*{Utilisation}

    Les commandes sont les suivantes:
    \begin{itemize}
        \item \textit{directeur creer <capacité\_musée> <capacité\_file>}\\
            \textit{directeur ouvrir}\\
            \textit{directeur fermer}\\
            \textit{directeur supprimer}\\
             effectue les différentes actions du directeur.
        \item \textit{controleur}\\
            met un controleur en poste.
        \item \textit{visiteur <durée\_de\_visite\_en\_ms>}\\
            envoie un visiteur visiter le musée.
        \item \textit{dump}\\
            affiche les valeurs courantes des sémaphores (utile pour le débugage).
    \end{itemize}

     \\

    La variable d'environnement \emph{DEBUG\_MUSEE} peut être utiliser pour influer sur l'affichage
    des commandes. Si la variable n'existe pas ou si elle vaut 0, alors les commandes n'affichent
    aucun message, à part les messages d'erreurs. Si \emph{DEBUG\_MUSEE} vaut 1, les programmes
    afficheront les etapes importantes des différents acteurs. Et si elle vaut 2 ou plus,
    les programmes afficheront tous les détails des actions des acteurs dont les étapes de synchornisation.

    \section*{Implementation}


    \subsection*{Mémoire partagée}

    Une structure \emph{musee} est placé en mémoire partagé à la création du musée, elle contient
    un booléen \emph{ouvert} indiquant si le musée est ouvert, et un entier \emph{capacite} indiquant
    la capacité maximum du musée.

    \subsection*{Sémaphores}

    Pour sychoniser les différents acteurs du musée, j'ai utilisé un ensemble de sept sémaphores:
    \begin{itemize}
        \item \emph{OUVERTURE\_SEM} initialisé à 0.\\
            Si le musée n'est pas ouvert les visiteurs et le controleur font un P (avec attente)
            en attendant l'ouverture du musée.\\
            Lors de l'ouverture du musée le directeur fait autant de V qu'il y a de processus
            qui attendent l'ouverture du musée.
        \item \emph{FILE\_SEM} initialisé au nombre maximum de places disponibles dans la file.\\
            Le visiteur fait un P (sans attente) au moment de rentrer dans le file,
            s'il y a de la place il y entre, sinon il s'en va.
            Une fois la valeur du sémaphore à 0 la file d'attente est pleine.\\
            Le controleur fait un V lorsqu'il fait entrer un visiteur dans le musée, ce qui libère
            une place dans la file d'attente.
        \item \emph{VISITE\_SEM} initialisé au nombre maximum de visiteus simultanés dans le musée.\\
            Le controleur fait un P (avec attente) lorqu'il fait entrer un visiteur dans le musée.
            Si le musée est plein, le controleur attend qu'une place se libère.\\
            Le visiteur fait un V lorsqu'il quitte le musée, ce qui libère une place dans le musée.
        \item \emph{ENTREE\_SEM} initialisé à 0.\\
            Le visiteur fait un P (avec attente) pour demander au controleur l'autorisation d'entrer
            dans le musée.\\
            Le controleur fait un V lorqu'il fait entrer un visiteur.
        \item \emph{TRAVAIL\_SEM} initialisé à 0.\\
            Le controleur fait un P (avec attente) pour attendre du travail.\\
            Le visiteur fait un V pour indiquer au controleur qu'il veux effectuer une action, soit
            entrer dans le musée, soit sortir du musée.
        \item \emph{VEUX\_ENTRER\_SEM} initialisé à 0.\\
            Le controleur fait un P (sans attente) pour voir si un visiteur veux entrer dans le musée.\\
            Le visiteur fait un V lorqu'il souhaite entrer dans le musée, pour prévenir le controleur.
        \item \emph{VEUX\_SORTIR\_SEM} initialisé à 0.\\
            Le controleur fait un P (sans attente) pour voir si un visiteur veux sortir du  musée.\\
            Le visiteur fait un V lorqu'il souhaite sortir du musée, pour prévenir le controleur.
    \end{itemize}
    \newpage

    \subsection*{Synchronisation}

    Le visiteur:
    \begin{itemize}
        \item Au lancement si le musée est fermé, il va faire un P de \emph{OUVERTURE\_SEM} qui va attendre
            que le musée ouvre.
        \item Il va ensuite faire un P (sans attente) de \emph{FILE\_SEM}, s'il arrive à entrer dans le
            sémaphore alors il est dans la file, sinon il s'en va et le processus se termine.
        \item S'il est dans la file d'attente, il fait un V de \emph{VEUX\_ENTRER\_SEM} puis un V de
            \emph{TRAVAIL\_SEM} pour signaler au controleur que celui ci doit le faire entrer dans le musée.\\
            Le V sont executés dans cette ordre car si V de \emph{TRAVAIL\_SEM} était executé avant V de
            \emph{VEUX\_ENTRER\_SEM} et que le quantum de temps processeur s'arrète entre les deux V.
            Alors le controleur ne peut pas savoir quelle travail il a à executer (faire entrer un
            visiteur ou le faire sortir). Dans ce cas il se remetterai en attente de travail sans
            prendre en compte la volonté de ce visiteur d'entrer.\\
            Mon implémentation protège de ce cas particulier en indiquant, avant même de réveiller
            le controleur, quelle action il aura à traiter à son réveil.
        \item Il demande ensuite l'autorisation d'entrer dans le musée en faisant un P (avec attente)
            de \emph{ENTREE\_SEM}.
        \item Une fois que le contrôleur l'a fait entrer dans le musée, le visiteur reste dans le
            musée, puis fait un V de \emph{VEUX\_SORTIR\_SEM} puis un V de \emph{TRAVAIL\_SEM}. Les V sont
            effectués dans cette ordre pour la même raison que lors de l'entrée dans le musée.
        \item Et pour finir le visiteur fait un V de \emph{VISITE\_SEM} pour indiquer sa sortie du musée,
            et libérer une place.
    \end{itemize}

     \\

    Le contrôleur:
    \begin{itemize}
        \item Au lancement il fait un P de \emph{OUVERTURE\_SEM} pour attendre que le musée ouvre.
        \item Ensuite, tant que le musée est ouvert:
            \begin{itemize}
                \item Il fait un P de \emph{TRAVAIL\_SEM} pour attendre du travail.
                \item Puis il fait un P (sans attente) de \emph{VEUX\_ENTRER\_SEM} pour voir si un
                    visiteur veux entrer dans le musée (si le visiteur a fait un V de
                    \emph{VEUX\_ENTRER\_SEM}).
                \item Si un visiteur veux entrer dans le musée, le contrôleur fait un P (avec
                    attente) de \emph{VISITE\_SEM} pour faire entrer le visiteur dans le musée
                    dès qu'il y a de la place. Il fait ensuite un V de \emph{FILE\_SEM} et un V de
                    \emph{ENTREE\_SEM}, pour indiquer qu'une place se libère dans le file d'attente
                    et pour débloquer le processus visiteur.
                \item Ensuite le contrôleur fait un P (sans attente) de \emph{VEUX\_SORTIR\_SEM} pour
                    afficher un message quand un visiteur sort du musée selon la valeur de la
                    variable d'environnement \emph{DEBUG\_MUSEE}.
            \end{itemize}
        \item Une fois que le musée est fermé, tant que la valeur du sémaphore \emph{VISITE\_SEM}
            est inférieur à la capacité maximal du musée (lue dans la mémoire partagée).
            \begin{itemize}
                \item Le contrôleur fait un P de \emph{TRAVAIL\_SEM} pour attendre du travail.
                \item Puis il fait un P (sans attente) de \emph{VEUX\_SORTIR\_SEM} pour voir si
                    un visiteur veux sortir et afficher un message en conséquence.
            \end{itemize}
        \item Une fois que tous les visiteur sont sortie du musée, le contrôleur se termine.
    \end{itemize}

    \newpage

    Le directeur:
    \begin{itemize}
        \item S'il est executé avec l'argument creer ou avec l'argument supprimer, aucun mecanisme
            de synchronisation n'intervient.
        \item Avec l'argument ouvrir: le directeur lis le nombre de processus qui attendent
            l'ouverture du musée, et fait V de \emph{OUVERTURE\_SEM} tant qu'il y a des
            processus qui attendent. Puis directeur si termine.
        \item Avec l'argument fermer: le directeur défini la valeur de \emph{OUVERTURE\_SEM} à
            zéro, et fait un V de \emph{TRAVAIL\_SEM} pour que le directeur qui est en attente de
            travail, remarque la fermeture du musée. Puis le directeur se termine.
    \end{itemize}

    \section*{Tests}

    Tous les scriptes de test fournis avec le sujet ont été éxécutés avec succès et sans
    fuite de mémoire.
    J'ai créé un scripte de test supplémentaire qui teste différents cas particuliers:
    \begin{itemize}
        \item La syntaxe \textit{directeur xxx xxx xxx}.
        \item La creation de plusieurs musées successifs dans suppression.
        \item La commande \textit{dump}.
        \item La fermeture d'un musée pendant que des visiteurs sont à l'intérieur.
    \end{itemize}

    \ \\

    La couverture de code lors des tests est de 93.28\%, seuls les cas où le système plante ne
    sont pas couvert par les tests.


    \section*{Conclusion}

    
    Ce TP a été très intéressant à réaliser, il m'a beaucoup fait réflechire
    et j'y ai passé de nombreuses heures. Il m'a beaucoup appris sur la
    synchronisation des processus et les différentes utilisations des
    sémaphores.

\end{document}
